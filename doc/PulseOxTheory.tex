\documentclass[11pt]{article}
\usepackage{geometry} % see geometry.pdf on how to lay out the page. There's lots.
\usepackage{hyperref}
\usepackage{graphicx}
\usepackage{gensymb}
\usepackage[affil-it]{authblk}
\usepackage[toc,page]{appendix}
\usepackage{pifont}
\usepackage{amsmath}
\usepackage{amssymb}
\usepackage{relsize}
\usepackage{draftwatermark}
\usepackage[mathscr]{eucal}

\SetWatermarkText{DRAFT}
\SetWatermarkScale{6}
\SetWatermarkLightness{0.95}

\renewcommand{\min}{\expandafter\,\operatorname*{min}}
\renewcommand{\max}{\expandafter\,\operatorname*{max}}

% \geometry{letter} % or letter or a5paper or ... etc
% \geometry{landscape} % rotated page geometry

% See the ``Article customise'' template for come common customisations

\title{An Explanation of Calibration-Free Pulse Oximetry}
\author{Robert L. Read
  \thanks{read.robert@gmail.com}
}
\affil{Founder, Public Invention, an educational non-profit.}


\date{\today}

%%% BEGIN DOCUMENT
\begin{document}

\maketitle

%% \tableofcontents

\section{Introduction}

This is an attempt to clarify the ideas presented by Chugh and Kaur\cite{ChughAndKaur2015}.

A valuble resource for understanding pulse oximetry in general is available at HowEquipmentWorks.com\cite{howPulseOx}.

The EWB-Austin Instrumentation group is hoping to teach the world how to construct a practical pulse oximeter inexpensively to provide
the medical benefits of pulse oximetry to parts of the world where it is not presently widespread. To do this, it must be understandable
and inexpensive.

We have breadboarded simple equipment and analyzed signals with an Arudino Uno, constituting a very simple pulse oximeter.

Chugh and Kaur\cite{ChughAndKaur2015} have presented a theoretical approach to constructing a pulse oximeter that does not require calibration.
If a pulse oximeter could be made withot calibration, this would be a large advantage for anyone in a developing country attempting to construct
and use one correctly.

\section{Review of Chugh and Kaur}

Chugh and Kaur\cite{ChughAndKaur2015} paper is concise, but not perfectly clear to us upon first reading. We here work through some of the
math they present in order to verify it and to verify that we correctly understand it.

In section $1.i$, the paper restates the Beer-Lambert law\cite{wiki:Beer-Lambert} as:

\begin{equation}
  A = \ln \frac{I_o}{I_t} = \varepsilon \cdot C \cdot L
  \end{equation}

where $A$ is the absorbance, $\varepsilon$ is the wavelength-dependent extinction\footnote{Note that Wikipedia\cite{wiki:MolarAttenuation} states the IUPAC discourages the
term {\emph extinction coefficient} in favor of {\emph molar attunation coefficient}.}
coeffecint, $C$ is the concentration of the absorbing material present in the
path and $L$ is the path length.


Using Wolframalpha.com to check this, we see that this formulation
(with a change of variable names)
is indeed just a restatement of the Beer-Lambert law as expressed by the
Wikipedia article\cite{wiki:Beer-Lambert}, although it seems that the computation
of the absorbance from the incident and transmitted light via 
logarithm should be, according to Wikipedia, base 10:

\begin{equation}
  A = \log_{10} \frac{I_o}{I_t} = \varepsilon \cdot C \cdot L
\end{equation}


Chugh and Kaur state the multi-species version of the Beer-Lambert law,
and then state the absorbance at the two distinct wavelenghts at which
oxygenated hemoglobin and deoxygenated hemoglobin differ maximally,
which they call $RED$ and $IR$, in terms of the concentrations of
two species (oxygenated ($C_{hbo}$) and
deoxygenated  ($C_{hb}$)).

\begin{equation}
  A_{RED} = (\varepsilon_{hbo(red)}\cdot C_{hbo} + \varepsilon_{hb(red)}\cdot C_{hb}) \cdot L
\end{equation}

\begin{equation}
  A_{IR} = (\varepsilon_{hbo(IR)}\cdot C_{hbo} + \varepsilon_{hb(IR)}\cdot C_{hb}) \cdot L
\end{equation}

The then define the ration $R$ as the ratio of the absorbance measured
at these wavelengths:

\begin{equation}
  R = \frac{A_{RED}}{A_{IR}}
  \end{equation}

Using straightforward algebra, R becomes independent of the path length $L$:
\begin{equation}
  R = \frac{(\varepsilon_{hbo(red)}\cdot C_{hbo} + \varepsilon_{hb(red)}\cdot C_{hb})}{(\varepsilon_{hbo(IR)}\cdot C_{hbo} + \varepsilon_{hb(IR)}\cdot C_{hb})}
\end{equation}

Using straightforward algebra, they rearrange this:
\begin{equation}
  \label{eq:chb}
  C_{hb} = C_{hbo} \frac{R \cdot \varepsilon_{hbo(IR)} - \varepsilon_{hbo(red)}}
  {\varepsilon_{hb(red)} - R \cdot \varepsilon_{hb(IR)} }
\end{equation}

Note: Chugh and Kaur use slightly different variable names than used at Wikipedia.
Possibly in this paper we will change the names yet again to make them more
conformant to that style.

They then define $SpO_2$ (periphereal oxygen saturation)
\cite{wiki:SpO2,wiki:oxygensaturation}:
\begin{equation}
  \label{eq:spo2}
  SpO_2 = \frac{C_{hbo}}{C_{hbo} + C_{hb}}
\end{equation}

Then the substitute \ref{eq:chb} into \ref{eq:spo2} and simplify:
\begin{equation}
  SpO_2 = \frac{100 (\varepsilon_{hbo(red)} - R \cdot \varepsilon_{hb(IR)})}
  {(\varepsilon_{hb(red)} - \varepsilon_{hbo(red)}) + R \cdot(\varepsilon_{hbo(IR)} - \varepsilon_{hb(IR)})}
\end{equation}

The then claim that R can be computed by taking the base-10 log of that AC 
component of the $RED$ and $IR$ signals.

I believe they paper suggests this to be the the difference between the the maximum
of the time-varying signal and the minimum of the time-varying signal.

(Only an electrical engineer would understand the term ``AC component'' to
mean the time-varying signal. I believe a better terminology is the
``pulsative signal''. This is the part of the signal remaining when the
unvarying signal is removed. In terms of the FFT, the very low frequencies
of the signal may be removed (those lower than the lowest human pulse, which
is about 0.5 Hz).)

However, the absolute difference between the greatest and least
transmittance (and inversely absorbance) is highly dependent on the
machinery and physical body measured.

Define the ``Pulsative Absorbance'' at frequency $\lambda$. 
\begin{equation}
  P_{\lambda} = \frac{\text{max of moving average of absorbance}}
  {\text{min of moving average of absorbance }}
\end{equation}

Where the absorbance is computed by the definition:
\begin{equation}
 A_{\lambda} = \log_{10} \frac{I_{\lambda o}}{I_{\lambda t}}   
\end{equation}

If we make the reasonable assumption that $I_t$ is proportional
to our our measured signal, then we can substitute and
remove the dependence ont he intensity of the transmitted light:
\begin{equation}
  P_{\lambda} = \frac{\max{A_{\lambda}}}
  {\min{A_{\lambda}}}
\end{equation}

So we can create the ratio of the blood-full absorbance
to the ratio of the blood-empty absorbance:
\begin{equation}
  P_{\lambda} = \frac{\max{A_{\lambda}}}
  {\min{A_{\lambda}}}
\end{equation}




Can it be that they mean the ratio of the pulsative part of the signal
to the non-pulsative part of the signal?  This is the change in
the absorbance.








\bibliographystyle{unsrt}
\bibliography{pulseox}


\end{document}




